%
% Chapter 6.2
%

\section*{6.2 Exponential Functions}

If \(b\) is a positive number and \(r\) is rational then,
\[ b^x= \lim_{r \to x}{b^r} \]

\begin{definition}
If \( b > 0 \) and \(b \neq 1 \), then \( f(x)=b^x \) is a continuous function with domain \( \mathbb{R} \) and range \( (0, \infty) \). In particular, \(b^x > 0\) for all \(x\). If \( 0 < b < 1 \), \(f(x) = b^x\) is a decreasing function; if \(b>1\), \(f\) is an increasing function. If \(a\), \(b > 0\) and \(x\), \(y \in \mathbb{R}\), then
\begin{enumerate}
\item \( b^{x+y}=b^x b^y \)
\item \( b^{x-y}=\frac{b^x}{b^y} \)
\item \( {(b^x)}^y=b^{xy} \)
\item \( {(ab)}^x = a^x b^x \)
\end{enumerate} 
\end{definition}
If \(b>1\), then \( \lim_{x \to \infty}{b^x} = \infty \) and \( \lim_{x \to -\infty}{b^x}=0 \)
\\
If \( 0<b<1 \), then \( \lim_{x \to \infty}{b^x} = 0 \) and \( \lim_{x \to -\infty}{b^x} = \infty \)

\subsection*{Definition of the Number \( e \)}
\(e\) is the number such that \[ \lim_{h \to 0}{\frac{e^h -1}{h}} =1 \]

\subsection*{Derivative of the Natural Exponential Function}
\[ \frac{d}{dx}(e^x)=e^x \]
\[ \frac{d}{dx}(e^u)=e^u \frac{du}{dx} \]

\begin{center}
\begin{tikzpicture}
\begin{axis}[
axis lines=middle,
    %grid=both,
    ymin=-1, ymax=6,
    xmax=3, xmin=-3,
    %xtick=\empty, ytick=\empty
]

% e^x function
\addplot [
    domain=-5:5,
    samples=100
]
{e^x};
\addlegendentry{ \( e^x \)}

% Add dot at (0,1)
\addplot [only marks] table { 
0  1 
};
 
\end{axis}
\end{tikzpicture}
\end{center}

\subsection*{Properties of the Natural Exponential Function}

The exponential function \( f(x)=e^x \) is an increasing continuous function with domain \( \mathbb{R} \) and range \( (0, \infty) \). Thus \( e^x > 0 \) for all \(x\). Also, 
\[ \lim_{x \to -\infty}{e^x} = 0 \quad \quad \lim_{x \to \infty}{e^x} = \infty \]
So the \(x\)-axis is a horizontal asymptote of \(f(x)=e^x\).

\subsection*{Integration of \( e \)}

\[ \int{e^x dx} = e^x + C \]
