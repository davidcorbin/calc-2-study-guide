%
% Chapter 11.6
%

\section*{11.6 Absolute Convergence and the Ratio and Root Tests}

A series \(\sum{a_n}\) is called \textbf{absolutely convergent} if the series of
absolute values \(\sum{|a_n|}\) is convergent.
\newline
A series \(\sum{a_n}\) is called \textbf{conditionally convergent} if it is
convergent but not absolutely convergent.
\newline
If a series \(\sum{a_n}\) is absolutely convergent, then it is convergent. 

\subsection*{The Ratio Test}

\begin{itemize}
    \item If \(\lim_{n \to \infty}{\big|\frac{a_{n+1}}{a_n}\big|}=L<1\), then
the series \(\sum_{n=1}^{\infty}{a_n}\) is absolutely convergent.
    \item If \(\lim_{n \to \infty}{\big|\frac{a_{n+1}}{a_n}\big|}=L>1\) or
\(\lim_{n \to \infty}{\big|\frac{a_{n+1}}{a_n}\big|}=\infty\), then the series
\(\sum_{n=1}^{\infty}{a_n}\) is divergent.
    \item If \(\lim_{n \to \infty}{\big|\frac{a_{n+1}}{a_n}\big|}=1\), the Ratio
Test is inconclusive.
\end{itemize}

\subsection*{The Root Test}

\begin{itemize}
    \item If \(\lim_{n \to \infty}{\sqrt[n]{|a_n|}} = L < 1\), then the series
\(\sum_{n=1}^{\infty}{a_n}\) is absolutely convergent.
    \item If \(\lim_{n \to \infty}{\sqrt[n]{|a_n|}} = L > 1\) or \(\lim_{n \to
\infty}{\sqrt[n]{|a_n|}} = \infty\), then the series
\(\sum_{n=1}^{\infty}{a_n}\) is divergent.
    \item If \(\lim_{n \to \infty}{\sqrt[n]{|a_n|}} = 1\), the root test is
inconclusive.
\end{itemize}
