%
% Chapter 7.2
%

\section*{7.2 Trigonometric Integrals}

\subsection*{Solving Integrals in the Form \( \int{\sin^m{x} \cos^n{x}} dx \)}

\begin{enumerate}

    \item If the power of cosine is odd, save one cosine factor and use \( \cos^2x=1- \sin^2x \) to express the remaining factors in terms of sine:

    \[ \int{\sin^m{x} \cos^{2k+1}x dx} = \int{\sin^m{x} {(\cos^2{x})}^k \cos{x} dx} = \int{\sin^m{x} {(1-\sin^{x}x)}^k \cos{x} dx} \]

    Then substitute \(u = \sin{x} \).

    \item If the power of sine is odd, save one sine factor and use \( \sin^2{x} = 1 - \cos^2{x} \) to express the remaining factors in terms of cosine:

    \[ \int{\sin^{2k+1}x \cos^{n}x dx} = \int{{(\sin^{2}x)}^k \cos^{n}x \sin{x} dx} = \int{{(1-\cos^2{x})}^k \cos^{n} \sin{x} dx} \] 

    Then substitute \(u=\sin{x}\).

    \item If the powers of both sine and cosine are even, use the half-angle identities

    \[ \sin^2{x} = \frac{1}{2}(1-\cos{2x}) \quad \cos^2{x} = \frac{1}{2} (1 + \cos{2x}) \]
    \[ \text{or} \] 
    \[ \sin{x}\cos{x} = \frac{1}{2}\sin{2x} \]
\end{enumerate}

\subsection*{Solving Integrals in the Form \(\int{\tan^m{x} \sec^n{x} dx} \)}

\begin{enumerate}
    \item If the power of secant is even, save a factor of \(\sec^2{x}\) and use \(\sec^2{x} = 1 + \tan^2{x}\) to express the remaining factors in terms of \(\tan{x}\). 
        \[ \int{\tan^m{x} \sec^{2k}x dx} = \int{\tan^{m}x {(\sec^2{x})}^{k-1}\sec^2{x}dx} = \int{\tan^{m}x{(\sec^2{x})}^{k-1}\sec^2{x} dx} \] 
    \item If the power of tangent is odd (\(m=2k+1\)), save a factor of \(\sec{x}\tan{x}\) and use \(\tan^2{x} = \sec^2{x} - 1\) to express the remaining factors in terms of \(\sec{x}\).
        \[ \int{\tan^{2k+1}x \sec^nx dx} = \int{{(\tan^2x)}^k \sec^{n-1}x \sec{x} \tan{x} dx} = \int{{(\sec^2{x} - 1)}^k \sec^{n-1}x \sec{x} \tan{x} dx} \]
\end{enumerate}

\subsection*{Antiderivative of Secant}

\[ \int{\sec{x} dx} = \ln{| \sec{x} + \tan{x} |} + C \]
