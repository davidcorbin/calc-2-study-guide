%
% Chapter 6.3
%

\section*{6.3 Logarithmic Functions}

The inverse of an exponential function is called a \textbf{logarithmic function} with base \(b\) and denoted by \( \log_b \).

\[ \log_b{x} = y \quad \Leftrightarrow \quad b^y = x \]

\subsection*{Graph of a Logarithmic Function}

\begin{center}
\begin{tikzpicture}
\begin{axis}[
axis lines=middle,
    %grid=both,
    ymin=-6, ymax=6,
    xmax=6, xmin=-6,
    %xtick=\empty, ytick=\empty
]

% b^x function
\addplot [
    dashed
]
{e^x};
\addlegendentry{ \( e^x \)}

% log(x) function
\addplot [
    samples=1000
]
{ln(x)};
\addlegendentry{ \( \ln{x} \)}

% Add dot at (0,1)
\addplot [only marks] table { 
0  1 
1  0
};
 
\end{axis}
\end{tikzpicture}
\end{center}

\[ \log_b (b^x)=x \text{ for every } x \in \mathbb{R} \]
\[ b^{\log_b x} = x \text{ for every } x > 0 \]

\begin{definition}
If \(b>1\), the function \(f(x)= \log_b x \) is a one-to-one, continuous, increasing function with domain \( (0, \infty) \) and range \( \mathbb{R} \). If \(x\), \(y > 0\) and \(r\) is any real number, then 
\begin{enumerate}
\item \( \log_b(xy) = \log_b x + \log_b y \)
\item \( \log_b(\frac{x}{y}) = \log_b x - \log_b y \)
\item \( \log_b(x^r) = r \log_b x \)
\end{enumerate} 
\end{definition}

\[ \text{If } b>1 \text{, then } \lim_{x \to \infty} \log_b x = \infty \quad \text{ and } \quad \lim_{x \to 0^+} \log_b x = - \infty \]

\subsection*{Natural Logarithms}

The logarithm with a base \(e\) is called the \textbf{natural logarithm} and special notation:
\[ \log_e x = \ln x \]

\[ \ln x = y \quad \Leftrightarrow \quad e^y = x \]
\[ \ln(e^x) = x \quad x \in \mathbb{R} \]

\[ \ln e = 1 \]

\subsubsection*{Change of Base Formula}

For any positive number \(b\) \((b \neq 1)\), we have 
\[ \log_b x = \frac{\ln x}{\ln b} \] 

\subsubsection*{Growth of the Natural Logarithm}

\[ \lim_{x \to \infty} \ln x = \infty \quad \quad \lim_{x \to 0^+} \ln x = x \infty \]
